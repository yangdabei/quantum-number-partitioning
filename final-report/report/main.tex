%\documentclass[11pt,draft]{article}
\documentclass[aps,twocolumn,preprintnumbers]{revtex4}
\usepackage[utf8]{inputenc}
\usepackage{fancyhdr}
\usepackage{natbib}
\usepackage{amsfonts, amsmath, amsthm, amssymb}
\usepackage{lipsum}
\usepackage[legalpaper, portrait, left=20mm, right=20mm, bottom=20mm, top=20mm]{geometry}
%\usepackage[parfill]{parskip}
\usepackage[autostyle]{csquotes}
\usepackage[shortlabels]{enumitem}
\usepackage{verbatim}
\usepackage{subfiles}
%\usepackage{breqn}
\usepackage{graphicx}
\usepackage{bm}

\usepackage{hyperref}
\hypersetup{
    colorlinks=true,
    linkcolor=black,
    filecolor=magenta,      
    urlcolor=blue,
    pdfpagemode=FullScreen,
    citecolor=blue
    }

\urlstyle{same}
\usepackage{appendix}
\usepackage{xcolor}
\usepackage{float}
%\usepackage[format=plain,
%            labelfont={bf,it},
%            textfont=it]{caption}
\usepackage{tikz}

\usepackage[labelfont={bf},textfont=it]{caption}
\usepackage{subcaption}


\newcommand{\lr}[3]{\!\left#1 #3 \right#2}
\newenvironment{solution}{\begin{proof}[Solution]}{\end{proof}}

\newtheorem{theorem}{Theorem}

\begin{document}
\title{Quantum Number Partitioning}
\author{Yangda Bei u7281660 \\ 3$^\text{rd}$ Year Mathematics \\ Advanced Studies Course SCNC3101}
\affiliation{ANU Research School of Physics,
            The Australian National University \\
            ANU Mathematical Sciences Institute, The Australian National University}




    \begin{abstract}
    \centering

        The number partition problem (NPP) is an NP-hard optimisation problem. We show how to formulate the NPP as a maximum cut (max-cut) problem of a complete, weighted graph, allowing us to apply the Quantum Approximate Optimisation Algorithm (QAOA) and its variants, multi-angle QAOA (MAQAOA) and eXpressive QAOA (XQAOA), to find optimal solutions. To benchmark the performances of the ansatzes on NPPs, we derive closed-form expressions at depth $p=1$ for NPP formulated as max-cut and compare it to the Goemans-Williamson (GW) algorithm for max-cut as well as classical NPP heuristics. Our results show that although XQAOA performs better than the GW algorithm, QAOA, and MAQAOA on complete, weighted graphs, classical NPP heuristics are more effective for solving NPPs for small problem sizes.
    \end{abstract}

\maketitle 


    \subfile{sections/introduction}

    \subfile{sections/preliminaries}
    
    \subfile{sections/problems}
    \subfile{sections/qaoa}

    \subfile{sections/simulations}

    \subfile{sections/discussion}

    \subfile{sections/futuredirections}

    

    \section*{Acknowledgements}

    I would like to thank Syed Assad for supervising this project, as well as Aritra Das and V Vijendran for their support. I would also like to thank Jeffrey Liang for his continual positivity and resilience, which allowed us to complete the project in high spirits.

    \bibliographystyle{unsrtnat}
    \bibliography{bibliography}



\end{document}