\section{Introduction}

With the current state of quantum computing hardware, known as Noisy Intermediate-Scale Quantum (NISQ) devices \cite{Preskill2018quantumcomputingin}, we are still quite far off from achieving general \emph{quantum supremacy} \cite{boixo2018characterizing}. That is, for quantum computers to outperform supercomputers on well-defined computational tasks, the fragility of coherent quantum systems must be overcome when building large-scale quantum computers. Such devices are believed to be able to address problems that are currently intractable for classical computers in areas such as computational chemistry \cite{mcardle2020quantum}, financial modelling \cite{orus2019quantum}, and machine learning \cite{dunjko2016quantum}. However, one particularly promising method to obtain quantum advantage on NISQ devices is the use of variational quantum algorithms (VQAs) \cite{cerezo2021variational}. Unlike traditional quantum algorithms, VQAs are a class of hybrid quantum-classical algorithms that seek to optimise an objective function by iteratively adjusting a set of quantum parameters in a quantum circuit, much like a classical neural network.

One of the most prominent VQAs is the Quantum Approximate Optimization Algorithm (QAOA), which has shown promise in solving combinatorial optimization problems \cite{farhi2014quantum}. QAOA has been studied extensively on the \emph{maximum cut problem} but has also been applied to practical applications such as the \emph{binary paint shop problem} \cite{streif2021beating} and the \emph{tail-assignment problem} \cite{vikstaal2020applying}.

In this study, we look at the \emph{number partitioning problem} (NPP), which has applications in multiprocessor scheduling \cite{sarkar1987partitioning}, voter manipulation \cite{walsh2009really}, and public key encryption schemes \cite{merkle1978hiding}. We present our findings on the effectiveness of quantum combinatorial optimisation methods on the NPP, which to our knowledge has not been done before. In particular, we evaluate the application of QAOA and its variants, multi-angle QAOA (MAQAOA) \cite{herrman2022multi} and eXpressive QAOA (XQAOA) \cite{vijendran2023expressive}, by reformulating NPPs as instances of weighted maximum cut problems. 

In Section \ref{sec:prelim}, we review some quantum computing foundations. In Section \ref{sec:problem}, we give an overview of the maximum cut problem and the NPP, and discuss the mapping of the problems to their respective Ising Hamiltonians. In Section \ref{sec:qaoa}, we review the Quantum Approximate Optimisation Algorithm (QAOA) and its variants. In Section \ref{sec:sim}, we show our results of maximum cut formulations of NPPs and provide a discussion on our findings in Section \ref{sec:disc}. We conclude with future directions in Section \ref{sec:fd}.

The repository for this project can be found \href{https://github.com/yangdabei/quantum-number-partitioning}{here}.
